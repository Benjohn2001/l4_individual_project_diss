% REMEMBER: You must not plagiarise anything in your report. Be extremely careful.

\documentclass{l4proj}
\graphicspath{ {images/} }
\usepackage{datetime}
\usepackage{placeins}
\usepackage{subfiles}
\usepackage{import}
\usepackage{url}
\newdateformat{MDY}{\monthname \space \THEDAY, \THEYEAR}

    
%
% put any additional packages here
%

\begin{document}

%==============================================================================
%% METADATA
\title{
    \includegraphics[width=200,height=200]{icon-clock-v2.png} \par
    {\huge Weasley Clock \par}
    {\Large Weasley clock inspired social media application \par}
}
\author{Ben Johnston - 2432411J}
\date{\MDY\today}

\maketitle

%==============================================================================
%% ABSTRACT
\begin{abstract}

    It can be said that humans rely on friendship, not only to prevent the feeling of loneliness, but for the positive mental attributions associated. A common problem however, is planning to socialise with these friends. Being unaware of a friend's status or location can often arise challenges that lead to a breakdown in plans. \vskip 0.5em

    This project aims to alleviate this struggle from the creation of a social media application graphically showcasing friends' statuses. The application concept is derived from the Weasley clock, a magical clock used by the Weasley family in the famous Harry Potter series. Friends' statuses are presented through a clock with user hands pointing to their status at a point in time. \vskip 0.5em

    The application was evaluated by potential users and received positive feedback with 84\% interested in concept, and 66\% stating they would potentially use the application. Application usability was also examined through the system usability scale with the app receiving an A+ grade, and positive feedback.
\end{abstract}

\def\consentname {Ben Johnston} % your full name
\def\consentdate {26 January 2023} % the date you agree
\educationalconsent


\tableofcontents
\newpage
\listoffigures

%==============================================================================
%% Notes on formatting
%==============================================================================
% The first page, abstract and table of contents are numbered using Roman numerals and are not
% included in the page count. 
%
% From now on pages are numbered
% using Arabic numerals. Therefore, immediately after the first call to \chapter we need the call
% \pagenumbering{arabic} and this should be called once only in the document. 
%
% Do not alter the bibliography style.
%
% The first Chapter should then be on page 1. You are allowed 40 pages for a 40 credit project and 30 pages for a 
% 20 credit report. This includes everything numbered in Arabic numerals (excluding front matter) up
% to but excluding the appendices and bibliography.
%
% You must not alter text size (it is currently 10pt) or alter margins or spacing.
%
%
%==================================================================================================================================
%
% IMPORTANT
% The chapter headings here are **suggestions**. You don't have to follow this model if
% it doesn't fit your project. Every project should have an introduction and conclusion,
% however. 
%
%==================================================================================================================================
\chapter{Introduction}

% reset page numbering. Don't remove this!
\pagenumbering{arabic} 
\import{chapters/}{introduction.tex}

%==================================================================================================================================
\chapter{Background}\label{bg}
\import{chapters/}{background.tex}

%==================================================================================================================================
\chapter{Analysis/Requirements}\label{anReq}
\import{chapters/}{requirements.tex}
%==================================================================================================================================
\chapter{Design}\label{des}
\import{chapters/}{design.tex}

%==================================================================================================================================
\chapter{Implementation}\label{imp}
\import{chapters/}{implementation.tex}

\chapter{Evaluation} \label{eval}
\import{chapters/}{evaluation.tex}

%==================================================================================================================================
\chapter{Conclusion}\label{conc}
\import{chapters/}{conclusion.tex}

%==================================================================================================================================
%
% 
%==================================================================================================================================
%  APPENDICES  
\begin{appendices}
\import{chapters/}{appendices.tex}
\end{appendices}


%==================================================================================================================================
%   BIBLIOGRAPHY   

% The bibliography style is abbrvnat
% The bibliography always appears last, after the appendices.
\bibliography{l4proj}
\bibliographystyle{plainurl}

% \bibliography{l4proj}
% \bibliographystyle{plain}


\end{document}
