This chapter will summarise the project as a whole whilst also addressing areas of the project that could be improved in future releases, a reflection on the project, and the personal development incurred.

\section{Summary}
This project stems from an initial idea proposed to create a digital adaptation of the Weasley clock used in the famous Harry Potter series. This was originally just a concept, an idea, that was later turned into the basis of a social media mobile application. From an initial design idea drawn out on paper, that then later progressed into a mobile application providing a rich experience to its users. This project aimed to alleviate the struggles faced when groups of friends attempt to make plans. Friends are unable to know whether their friends are available, or whether they are at work for example. This app solves this issue by providing the user with an easy to interpret, graphical representation of their friend's status.\newline\newline
The project started with an analysis of the problem, to help devise a sensible plan for completing the project goals. During this analysis stage techniques such as user personas, user stories, requirement gathering, and prioritisation were used to form a set of achievable goals that satisfied the potential user needs, and goals. Upon completion of the project, all the prioritised functional, and non-functional requirements were met with only low priority requirements being left for future releases. \newline\newline  
With a project plan, and requirements in place an initial design was created, and then turned into a high-quality design prototype using Figma. There were many considerations during the design process such as the Web Content Accessibility Guidelines (WCAG), and Nielsen's heuristics to ensure that the application was not only aesthetically pleasing but usable with a high level of accessibility.\newline\newline
These designs formed the basis of the implementation with the application built upon these designs with usability considerations in mind. The implementation phase involved the initial setup, and maintenance of the Firebase services and React Native application code. Firebase's realtime database was used to store and manage all the data for the application, their cloud storage was used to store all images, and their authentication was used to control access to these resources, and user accounts. Expo was used to streamline implementation by providing quick build times, and a rich developer experience without the need for build tools. These technologies were all integrated with the JavaScript application code created with the React Native JavaScript framework. \newline\newline
An evaluation was then conducted on this completed mobile application gathering results and feedback on the completed product. This evaluation was split into an online and in-person experiment allowing for the collection of a suitable number of participant responses. The application received very positive feedback with the majority of participants interested in the concept, and application itself. System usability was measured during the in-person experiment by the system usability scale with the application receiving an A+ grade of 84.38. Users found the application design consistent, and easy to navigate which was a main goal of the project. Overall, participants responded positively to both experiments providing positive feedback and areas for improvement that can be addressed in future releases.  


\section{Future Work}
This section covers areas of the application that given more time, could be improved.
\subsection*{Increase Test Coverage}
Currently, the testing of the application is not fully complete as discussed in \ref{testChall}. All components have a high level of coverage with 84.31\% of lines of component code covered by unit, and snapshot testing. However, as previously discussed, testing the application code, and Firebase services, requires either the mocking of all Firebase services, and functions used in the application, or the use of the Firebase local emulator suite. Although both of these options were not feasible within the time constraints of this project, this would be an area that given more time, would be good to improve.  

\subsection*{Theme Changes}
A potential improvement suggested by multiple participants during the evaluation was the ability to change the theme of the application and have integration with light, and dark modes. This was a requirement prioritised as `won't have' in \ref{non-functional}, this meant it was an area considered, but not vital for project success in this release. This feature would not only increase user satisfaction but also add to the accessibility of the application with the integration of dark mode.

\subsection*{Push Notifications}
The integration of push notifications was also a suggested improvement by evaluation participants, and a `could have' prioritised requirement. As discussed in research by Atilla Wohllebe \cite{pushNoti}, push notifications when used correctly can drive user engagement and help add value to the application. However, when used incorrectly push notifications can disturb users and lead to a loss in user satisfaction. The integration of push notifications if done correctly could add value to this application, in a future release this could be a valuable area to investigate.

\subsection*{Real-World Location \& Animations}
To improve the user experience, the idea of including a user's real-world location, and animations to the clock were considered, and listed as `won't have' requirements. The integration of a user's real-world location with the application could be implemented in many forms, a user's status could be updated simply by the user being present in a specific location, or even used to show where friends are on a graphical map as well as a clock. Animations for a user changing their status or adding a member could be shown by the hands moving for example. This would be a purely cosmetic improvement but would add to the user experience.

\section{Reflection}

Upon reflection, this project was a great learning experience, from not only the developer's standpoint but also personally. The experience gained from taking on a project requiring many skills not only computing related, but personal, was invaluable. From coordinating and attending meetings with a project supervisor to implementing a graphical clock this project has facilitated the expansion of knowledge, and personal skill set to great lengths. In the end, this project was a great success with the main project goals, and requirements being met to a high standard. Some areas would be great to further improve upon in future releases that could not be implemented due to the time constraints of the project in this release, but hopefully can be in the future. Mobile application development is an area that has not been covered by the University courses taken, which proved difficult when first implementing the application. However, this is now an area where great confidence and interest have grown, helping uncover a new line of work that could be investigated in future employment.  