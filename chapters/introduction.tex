The chapter will discuss the motivation and goals of the project along with providing an overview of the paper.

\section{Overview}

In this dissertation, the full software development process of creating the application `Weasley Clock' will be discussed. This will include the problem analysis and requirement gathering stage, the design and implementation phase, and the process of evaluating the project. This project was undertaken as part of the Level 4 Computer Science program at the University of Glasgow. This project was completed by Ben Johnston and overseen by Dr Jeremy Singer, as project supervisor.

\section{Motivation}
Since the beginning of time humans have engaged in friendships, creating groups of friends to socialise with for support, or even just to not feel lonely as studied by Apostolou et al. \cite{whyFriends}. A common problem amongst humans in this digital age however, is planning to socialise with these friends. Being unaware of a friend's status or location often presents challenges when planning, and likely leads to a breakdown of plans.  \newline\newline
Social media usage is now undoubtedly booming, a study conducted in 2015 reported that 65\% of adults in America now use social media \cite{socialMediaUsage}. Social media is a powerful way of presenting information to users quickly, and easily through the use of electronic devices. Results from a study by Subrahmanyam et al. \cite{SUBRAHMANYAM2008420} show that humans are now using these social media applications to connect, and reconnect with their peers.


\section{Goals}
This projects aim is to create a social media application that can be used by groups of friends to communicate their status. This status would be communicated graphically through a clock with sectors representing a status, and user clock hands pointing to the corresponding status sector. As this will be a software engineering project the main aim will be to create a robust, efficient, and reliable software product. This application will be used by people of varying technology backgrounds, ages, and abilities so must be designed to be easy to use, quick to learn, and intuitive. An evaluation shall be conducted to measure the success of the project regarding usability, design, and general application feedback.

\section{Outline}

This dissertation will cover the process undertaken creating the Weasley clock mobile application. The background research conducted into similar applications, and concepts that formed a basis for this project will be discussed in \textbf{Chapter \ref{bg}, Background}. The process of eliciting requirements through requirement gathering methods, and prioritisation along with the analysis of the problem will be covered in \textbf{Chapter \ref{anReq}, Analysis/Requirements}. The design phase that included the initial design decisions, wire frames, and architectures will then be discussed in \textbf{Chapter \ref{des}, Design}. The process of then implementing these designs along with the justification of technologies used with their advantages, and challenges presented discussed in \textbf{Chapter \ref{imp}, Implementation}. The evaluation process including testing, evaluation experiments, and the results gathered will then be discussed in \textbf{Chapter \ref{eval}, Evaluation}. The dissertation will conclude with a final summary of the project, the future work to be completed, and a personal reflection in final \textbf{Chapter \ref{conc}, Conclusion}.