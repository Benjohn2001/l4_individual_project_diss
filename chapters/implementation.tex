\usepackage{listings}
This chapter will discuss the process of implementing the application. The justification of the technologies used along with the benefits, and, challenges they potentially presented will be discussed here.

\section{Version Control}
Version control is required to have been used throughout the duration of the project. Version control allows us to keep track and manage changes made to source code and other software artefacts. Good utilization of version control is critical to the success of a project, and helps to keep a project organised.
\subsection{GitHub \& Git}
GitHub \cite{github} is a platform widely used throughout the industry to store software projects online. GitHub allows users to create Git repositories \cite{git} that contain the files used within a project along with their version history. This combination is widely used within the industry due to the distributed version control and the various tools offered by GitHub to enhance the developer experience. Both these tools were very familiar so made this a simple choice. 
\subsubsection{Feature Branching}
An advantage of using a distributed version control system such as Git is how easy branching is due to the developer being able to create multiple branches at a local level. Feature branching was used throughout this project (see \ref{fig:branching}), with new features being developed in local feature branches and then merged back with the main branch upon completion.  
\begin{figure}[!htbp]
    \centering
    \begin{subfigure}[b]{0.90\textwidth}
        \frame{\includegraphics[width=\textwidth]{featureBranch.png}}
    \end{subfigure}
    \caption{Example of feature branching used in project}
    \label{fig:branching}
\end{figure}
\subsubsection{Issue Tracking}
The GitHub issue tracker was used to keep track of what tasks were still to be completed, along with any bugs that needed to be fixed. GitHub allows you to apply labels to an issue helping you to separate between Firebase and design issues for example (see \ref{fig:issues}).
\begin{figure}[!htbp]
    \centering
    \begin{subfigure}[b]{0.90\textwidth}
        \frame{\includegraphics[width=\textwidth]{issues.png}}
    \end{subfigure}
    \caption{Examples of issues created in GitHub's issue tracker}
    \label{fig:issues}
\end{figure}
\subsubsection{Actions}
GitHub's actions is a continuous integration and continuous delivery tool that allows you to create workflows that are triggered on specific behaviour. A workflow that built and published the application upon a push to the main branch was created so that the newest version of the app was always accessible on my mobile device. A testing workflow was also created with any pushes to the tests, or main branch running the test suite along with code formatting and style checks (see \ref{CodeF&S}).
\section{Technologies}
A comparative study \cite{compStudy} was performed in the early stages to help decide what technologies were best suited to the task. The technologies used throughout the project will be discussed in this section.
\subsection{React Native}
When creating a mobile application you currently have two options, native applications for each operating system, or, using frameworks that allow a single code base to create applications for both operating systems. While there are many positives from creating native apps such as being able to utilize many platform specific features, the extra development time to create two consistent, but separate apps was simply infeasible within the time constraints. React Native \cite{reactnative} allows you to create whole applications using only JavaScript \cite{js} in a singular code base. This was chosen over other alternative due to the component nature and flexibility provided by the framework. 
\subsubsection{Expo}
Solely React Native development does however require some native coding and the presence of build tools for both operating systems which was an issue. This meant that both Android Studio and XCode would both have to be installed on the device used for developing the application which was an issue for storage space. Due to no native coding being required in the project the choice to use Expo \cite{expo}, a framework used to help ease building React Native applications was made. Expo requires no native coding or build tools allowing you to simply run a command in the terminal to be able to access the application on your device or emulator.\begin{lstlisting}[language=bash, caption=To build and start applications with expo for iOS or Android]
  $ expo start --ios
  $ expo start --android
\end{lstlisting}
\subsection{Tailwind CSS}
When creating a user interface there are several CSS frameworks and UI kits that can be used in tandem, or, independently of standard CSS style sheets. Tailwind CSS \cite{tailwind} is a CSS framework that allows the developer to create reusable styled components simply by using utility classes in the className property.\begin{lstlisting}[language=JavaScript, caption={Creates a dark purple, rounded view with height 8 units, width 10 units, and, a margin left and right of 1.5 units}]
  <View className="bg-darkerPurple h-8 w-10 rounded-full mx-1.5">
\end{lstlisting}
Tailwind does not include styled components giving full control to the developer over the design. This approach was chosen over a UI kit of created components to allow the design to be unique and flexible to the project requirements. NativeWind \cite{nativewind} is a package used to process the CSS from the tailwind class names into StyleSheet objects, that can then be understood by ReactNative. The use of this technology enhanced the user interface and developer experience by allowing full flexibility over design, and, an easier alternative to traditional CSS style sheets.   
\subsection{Code Formatting \& Style} \label{CodeF&S}
\subsubsection{ESLint}
\subsubsection{Prettier}
\subsection{Firebase}
\subsubsection{Authentication}
\subsubsection{Storage}
\subsubsection{Realtime Database}
\section{Challenges}
